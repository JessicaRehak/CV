%______________________________________________________________________________________________________________________
% @brief    LaTeX2e Resume for Kathryn D. Huff

\documentclass[margin,line]{resume}
\usepackage{bibentry}
\usepackage[pdftex, pdfauthor={J. S. Rehak}, pdftitle={J.S. Rehak CV}]{hyperref}
\hypersetup{
    colorlinks=true,
    linkcolor=blue,
    filecolor=magenta,
    urlcolor=cyan,
  }
    
\newcommand{\Cpp}[1][]{\textrm{C\nolinebreak[4]\hspace{-.05em}\raisebox{.4ex}{\tiny\bfseries++}#1}}
\urlstyle{tt}
\usepackage{xspace}
%\textheight=10.75in
\newcommand{\Cyclus}{\textsc{Cyclus}\xspace}%
\newcounter{daggerfootnote}
\newcommand*{\daggerfootnote}[1]{%
    \setcounter{daggerfootnote}{\value{footnote}}%
    \renewcommand*{\thefootnote}{\fnsymbol{footnote}}%
    \footnote[2]{#1}%
    \setcounter{footnote}{\value{daggerfootnote}}%
    \renewcommand*{\thefootnote}{\arabic{footnote}}%
}
\newcommand{\dagfoot}{${}^\dagger$}
%______________________________________________________________________________________________________________________
\begin{document}
\name{\Large Jessica S. Rehak}
\begin{resume}
    % Contact Information
  \section{\mysidestyle Contact\\Information}
  phone: (401) 573-7417 $\cdot$ email: \verb`jessica.s.rehak@gmail.com`\newline
  github: \url{https://github.com/JessicaRehak}\hfill \\

    %__________________________________________________________________________________________________________________
    %Resume Objective
    %\section{\mysidestyle Objective}
    %          Seeking research and practical opportunities in
    %           nuclear engineering and scientific computation to
    %           support the next generation of reliable and safe
    %           nuclear energy.%
    %__________________________________________________________________________________________________________________
    % Research Interests
    % \section{\mysidestyle Research\\Interests}
    %             Advanced nuclear reactors and fuel cycles, multi-physics
    %             simulation, nuclear fuel cycle analysis,
    %             scientific computation.
    %             %
    %__________________________________________________________________________________________________________________
    % Academic Appointments
    %           \vspace{-2mm}
    %\section{\mysidestyle Academic\\Appointments}
    %           \vspace{-2mm}\\\vspace{-3mm}%
    %__________________________________________________________________________________________________________________
    % Education
    \section{\mysidestyle PhD}
    \textbf{University of California, Berkeley}, \textsc{Nuclear
      Engineering}\hfill \textbf{Summer 2021}\vspace{-3mm}\\\vspace{-1mm}%
    \begin{list2}
        \item[] A Novel Tool for the Assessment and Validation of Acceleration Methods for Solving the Neutron Transport Equation
        \item[] Advisor:  Professor Rachel N. Slaybaugh
    \end{list2}\vspace{-1.5mm}
    \section{\mysidestyle MS}
    \textbf{University of California, Berkeley}, \textsc{Nuclear
      Engineering}\hfill\textbf{Spring 2017}\vspace{-3mm}\\\vspace{-1mm}%
    \begin{list2}
        \item[] Implementation of Weighted Delta-tracking with
          scattering in Serpent 2
        \end{list2}\vspace{-1.5mm}
    \section{\mysidestyle MEM}
    \textbf{Old Dominion University -- Norfolk VA},
    \textsc{Engineering Management}%\daggerfootnote{Awarded or issued under the name Joshua. S. Rehak}\hfill\textbf{Fall 2015}\vspace{-3mm}\\\vspace{-1mm}%
    \section{\mysidestyle BS}
    \textbf{University of Maryland, College Park},
    \textsc{Physics, Astronomy}\hfill\textbf{Spring 2007}\vspace{-3mm}\\\vspace{-1mm}%
\vspace{-1.5mm}
    %__________________________________________________________________________________________________________________
    % Work & Research Experience
    \section{\mysidestyle Work \& Research \\Experience}
    \textbf{Kairos Energy}, Alameda, CA \\
                \textsl{Reactor Analyist (Engineer III)} \hfill
                \textbf{September 2021 -- October 2023} \\
                \textsl{Supervisor: Nader Satvat} -- \verb`satvat@kairospower.com`
                \begin{list2}
                \item[] Developed a series of Python 3-based deployable libraries designed to work in concert to support the rapid iterative design and evaluation of novel pebble bed reactors. Some of the features included:
                \begin{list2}
                \item conversion of material specifications provided in various formats into a standardized and shared elemental format,
                \item generation of a full pebble bed core model from a collection of user-defined yaml files, stored in a comprehensive Python 3 object complete with materials assigned to a polygon-based cross-sectional geometry,
                \item and generation of a core input file for the Serpent 2 Monte Carlo code, converting arbitrary rectangular and triangular polygons, as well as other features, into appropriate Serpent surfaces and cell definitions; evaluation of complex multi-point surfaces into a series of polygons to describe highly-detailed features or core regions; automatic determination and elimination of overlapping surfaces; and handling of regions with multiple layered universes, and distributed pebble-bed fuel materials.
                \end{list2}
                \end{list2}
    \textbf{University of California, Berkeley}, Berkeley, CA\\
                \textsl{Graduate Student Researcher} \hfill
                \textbf{Fall 2015 -- Summer 2021}\\
                \textsl{Advisor: Professor Rachel N. Slaybaugh} -- \verb`slaybaugh@berkeley.edu`
                \begin{list2}
                \item[]                 Developed a novel finite-element-based code for the implementation and
                assessment of acceleration methods for deterministic
                solves of the transport
                equation that leveraged modern \Cpp{} features, documentation systems, and testing frameworks.
                \end{list2}

    \textbf{The Idaho National Laboratory}, Idaho Falls, ID \\
                \textsl{Student Intern - Reactor physics group} \hfill
                \textbf{Summer 2016}\\
                \textsl{Advisor: Dr. Mark DeHart} --
                \verb`mark.dehart@inl.gov`
                \begin{list2}
                \item[] Implemented a novel delta-tracking algorithm for the Serpent 2 Monte Carlo code.
                \end{list2}
    \textbf{United States Navy} \\
    \textsl{Submarine Officer -- Honorably discharged as a lieutenant (O-3)} \hfill \textbf{2008 -- Fall 2015}
    \\\vspace{-3mm}
    \begin{list2}
    \item Coordinated submarine operations and international
      participation for the Rim of the Pacific 2014 naval exercise
      involving 23 nations, 46 ships and six submarines.
    \item Supported two six-month deployments while qualified Officer
      of the Deck and Engineering Officer of the Watch on Los Angeles class submarines.
    \item Certified for assignment as Engineer Officer in charge of a Naval Nuclear Propulsion Plant.
    \item Led divisions responsible for the maintenance and operation of reactor plant instrumentation, radiological controls, and water chemical analysis.
      \item TS/SCI security clearance (single scope background investigation).
          \end{list2}
                
    \section{\mysidestyle Publications \\ \& Proceedings}
    %\section{\mysidestyle Refereed\\Journal\\Publications} % also, vspace below
    \begin{bibsection}
    \item \textbf{Rehak, J.S.}, Slaybaugh, R.N. ``Assessing the
      Effectiveness of Acceleration Methods for Deterministic Neutron
      Transport Solvers'' \textbf{Transactions of the American Nuclear
        Society} Volume 122. \url{https://doi.org/10.13182/T122-32383} June 2020.\dagfoot
      \item \textbf{Rehak, J.S.}, Kerby, L.M., DeHart, M.D., Slaybaugh, R.N.
              ``Weighted delta-tracking in scattering media''
              \textbf{Nuclear Engineering and Design} Volume 342.
              \url{https://doi.org/10.1016/j.nucengdes.2018.12.006}. December 2018.\dagfoot
            \item \textbf{Rehak, J.S.}, Kerby, L.M., DeHart, M.D., Slaybaugh, R.N., Lepp\"{a}nen, J.
              ``Implementation of Weighted Delta-Tracking with Scattering in the Serpent 2 Monte Carlo Code''
              \textbf{Transactions of the American Nuclear Society}
              Volume
              116. \url{https://escholarship.org/uc/item/6bg1s71k} June 2017.\dagfoot
      \end{bibsection}
      %\vspace{2mm} % add this back if you return to Refereed Journal..
      %\section{\mysidestyle Submitted}
      %\begin{bibenum}
      %\end{bibenum}
                

    %__________________________________________________________________________________________________________________
    % Honors and Awards
    \section{\mysidestyle Honors and\\Awards}
    Department of Nuclear Engineering Graduate Fellowship \hfill \textbf{2015 -- 2018}\\%
    Navy and Marine Corps Commendation Medal \hfill \textbf{August 2015}\\\vspace{-3.5mm}
        \begin{list2}
        \item[] For exceptional service as Submarine Force Exercise Officer and Submarine Watch Officer at Commander Submarine Forces Pacific
        \end{list2}\vspace{-3mm}
    Navy and Marine Corps Achievement Medal \hfill \textbf{August 2015}\\\vspace{-3.5mm}
        \begin{list2}
        \item[] For coordination and execution of submarine operations for the Rim of the Pacific 2014 exercise
        \end{list2}\vspace{-3mm}
    Navy and Marine Corps Achievement Medal \hfill \textbf{June 2013}\\\vspace{-3.5mm}
        \begin{list2}
        \item[] For service as a division officer on USS JACKSONVILLE (SSN-699) and successful completion of two six-month deployments and an extended dry-dock maintenance period.
        \end{list2}\vspace{-3mm}
    Navy and Marine Corps Achievement Medal \hfill \textbf{April 2013}\\\vspace{-3.5mm}
        \begin{list2}
        \item[] For service as Chemistry/Radiological Assistant during an eight month dry-dock period.
        \end{list2}\vspace{-3mm}
    Navy and Marine Corps Achievement Medal \hfill \textbf{January 2011}\\\vspace{-3.5mm}
        \begin{list2}
        \item[] For service as Reactor Control Assistant during a six-month deployment and Operational Reactor Safeguards Exam
        \end{list2}             \vspace{-3mm}           

        
        % __________________________________________________________________________________________________________________
        \section{\mysidestyle Code Development}
        \textbf{Bay Area Radiation Transport (BART)} \hfill
        \url{https://github.com/SlaybaughLab/BART} \\
        A finite-element-based transport solver that
                  supports 1/2/3D and MPI, based on the \href{https://www.dealii.org/}{deal.II}
                  finite element library.
                \begin{list2}
                \item Designed for developer end-users for maximum
                  modification and support of methods analysis and
                  implementation.
                \item Designed to support reproducibility,
                  portability, and testing in codes. utilizes
                  continuous integration, code coverage, and Docker
                  containers.
                \item Uses a novel protocol-buffer format for materials.
                \end{list2}
    % Computer Skills
    \section{\mysidestyle Scientific\\Computing\\Skills}
                \textbf{Languages} \dotfill  Python 3, \Cpp{20}, bash\vspace{1mm}\\%
                \textbf{Build Systems} \dotfill setuptools, make, CMake \vspace{1mm}\\%
                \textbf{Testing} \dotfill pytest, GoogleTest, GoogleMock, continuous integration, code coverage\vspace{1mm}\\%
                \textbf{Version Control} \dotfill git, github\vspace{1mm}\\%
                \textbf{Other} \dotfill sphinx, Doxygen, \LaTeX, 
                Protocol Buffers, Jupyter, Docker \vspace{1mm}%

    %__________________________________________________________________________________________________________________
    % References

%\section{\mysidestyle References}
%\texsl{Available upon request}

%______________________________________________________________________________________________________________________


    %__________________________________________________________________________________________________________________
\end{resume}

\end{document}


%______________________________________________________________________________________________________________________
% EOF


%%% Local Variables:
%%% mode: latex
%%% TeX-master: t
%%% End:
